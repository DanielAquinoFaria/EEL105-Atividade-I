% Template for articles for CCR (Computational Communication Research)

% This file contains information such as author, title, etc.
% Edit the body.tex file to add the contents (or change the % The article body
% Also see main.tex for information such as authors, title, etc
% See https://www.overleaf.com/learn/ for some great resources on learning latex

\section{Introdução}

Texto desenvolvido como requisito para a atividade I do Curso: Processos de Fabricação de dispositivos Eletrônicos, no mesmo é comentado sobre a extração de silício como matéria prima no Brasil, e sobre as empresas presentes no mesmo.

\section{Brasil e os Semicondutores}

A produção de semicondutores é um mercado global em constante crescimento e evolução. No Brasil, a indústria de semicondutores é relativamente nova, mas vem ganhando destaque nos últimos anos, com investimentos em pesquisa, desenvolvimento e produção de chips eletrônicos. A produção de semicondutores é um setor estratégico para a economia nacional, com aplicações em diversos setores, desde a indústria automotiva até a produção de dispositivos móveis e computadores.\\
A matéria-prima essencial para a produção de semicondutores é o silício. No Brasil, a extração de silício é realizada principalmente em Minas Gerais e no Pará, que contam com minas de alta qualidade e tecnologia avançada de produção. \\
Uma das principais entidades do setor de semicondutores no Brasil é a ABISEMI (Associação Brasileira da Indústria de Semicondutores), que tem como objetivo fomentar o crescimento e a competitividade da indústria nacional de semicondutores. A ABISEMI oferece suporte para empresas do setor, além de promover eventos e fóruns de discussão sobre a indústria de semicondutores no Brasil.\\
A CEITEC (Centro Nacional de Tecnologia Eletrônica Avançada) é uma empresa pública criada em 2008 para promover a produção nacional de semicondutores. A empresa tem como objetivo desenvolver tecnologias de ponta e incentivar a inovação tecnológica no país, com projetos de pesquisa e desenvolvimento de circuitos integrados para diversas aplicações, como identificação por radiofrequência (RFID) e smartcards.\\
A Intel, uma das maiores empresas de tecnologia do mundo, também tem investido em produção de semicondutores no Brasil. Em 2012, a empresa inaugurou uma fábrica em Santa Rita do Sapucaí, em Minas Gerais, para a produção de chips eletrônicos. A fábrica da Intel no Brasil é a primeira da empresa na América Latina e representa um importante avanço na consolidação da indústria nacional de semicondutores.\\
A Samsung, outra gigante da tecnologia, também tem investido no mercado de semicondutores no Brasil. A empresa possui uma fábrica em Campinas, São Paulo, que produz chips para dispositivos móveis e memórias flash NAND. A Samsung tem se destacado pela sua capacidade de inovação e pela qualidade de seus produtos, contribuindo para a competitividade da indústria nacional de semicondutores.\\
A Motorola, empresa americana de telecomunicações, também tem histórico de investimentos em produção de semicondutores no Brasil. Em 1998, a empresa inaugurou uma fábrica em Jaguariúna, São Paulo, para a produção de chips eletrônicos e outros componentes. Embora a empresa tenha enfrentado dificuldades financeiras nos últimos anos, seu histórico de investimentos e inovação na indústria de semicondutores no Brasil.\\
Com o crescente avanço tecnológico e a constante demanda por chips eletrônicos, a indústria de semicondutores é cada vez mais estratégica para o desenvolvimento econômico e tecnológico do Brasil. Investimentos em pesquisa, desenvolvimento e produção de semicondutores são essenciais para aumentar a competitividade da indústria nacional e promover a inovação tecnológica no país.\\
No entanto, a produção de semicondutores é um mercado altamente competitivo e dinâmico, com inúmeras empresas disputando espaço em um cenário global. Para manter a competitividade, o Brasil precisa continuar investindo em pesquisa e desenvolvimento de tecnologias de ponta, além de fomentar o empreendedorismo e a inovação em todas as etapas da cadeia de produção de semicondutores.\\
A indústria de semicondutores no Brasil tem muito potencial de crescimento, mas também enfrenta desafios, como a alta carga tributária, a falta de incentivos fiscais e a falta de mão de obra especializada. É importante que o governo e as empresas do setor trabalhem juntos para superar esses desafios e garantir que a indústria de semicondutores no Brasil continue a crescer e a contribuir para o desenvolvimento econômico e tecnológico do país.\\
Em resumo, a produção de semicondutores é um setor estratégico e em constante crescimento no Brasil, com empresas nacionais e internacionais investindo em pesquisa, desenvolvimento e produção de chips eletrônicos. Com o apoio do governo e da sociedade, a indústria de semicondutores tem potencial para se tornar uma importante força impulsionadora da economia e da inovação tecnológica no país.




% \begin{figure}[b]
%     \centering
%     \includegraphics[width=.6\textwidth]{aup.png}
%     \caption{Caption of your figure}
%     \label{fig:my_label}
% \end{figure}



  

 below)

\documentclass{ccr}

% Note - lines starting with a % are completely ignored by latex and function as comments
% See https://www.overleaf.com/read/hmwdsgcqkxrd for a complete example article

% The first part of the latex document contains metadata

% Regular metadata (title, authors etc)
\title{Atividade I}
\authorsnames{Daniel Florencio de Aquino Faria}
% Short author list for footer


% Note that you need to provide as many affiliations as authors
% If multiple authors have the same affiliation, please copy that record as needed
\authorsaffiliations{
  {Universidade Federal do ABC}, 
}

% You can add or rename the bibliography file(s) here
% Note that you can exported them from endnote or zotero directly and upload them to overleaf
\addbibresource{bibliography.bib}


% The following information is provided by the journal when moving to production
\volume{?}
\pubnumber{?}
\pubyear{20??}
\firstpage{1}
\doi{10.5117/CCR????}

% some packages that are generally useful when writing articles in latex
\usepackage{tabularx}  % for full-width tables
\usepackage{booktabs}  % for nicer horizontal lines in tables
\usepackage[mode=text]{siunitx} % for centering columns on the decimal mark
\usepackage{graphicx} % for including figures
\usepackage{csquotes}\MakeOuterQuote{"}  % to allow "double quotes" instead of ``double quote''
\usepackage[hidelinks]{hyperref} % for URLs and other hyperlinks


\begin{document}
\maketitle

% The article body
% Also see main.tex for information such as authors, title, etc
% See https://www.overleaf.com/learn/ for some great resources on learning latex

\section{Introdução}

Texto desenvolvido como requisito para a atividade I do Curso: Processos de Fabricação de dispositivos Eletrônicos, no mesmo é comentado sobre a extração de silício como matéria prima no Brasil, e sobre as empresas presentes no mesmo.

\section{Brasil e os Semicondutores}

A produção de semicondutores é um mercado global em constante crescimento e evolução. No Brasil, a indústria de semicondutores é relativamente nova, mas vem ganhando destaque nos últimos anos, com investimentos em pesquisa, desenvolvimento e produção de chips eletrônicos. A produção de semicondutores é um setor estratégico para a economia nacional, com aplicações em diversos setores, desde a indústria automotiva até a produção de dispositivos móveis e computadores.\\
A matéria-prima essencial para a produção de semicondutores é o silício. No Brasil, a extração de silício é realizada principalmente em Minas Gerais e no Pará, que contam com minas de alta qualidade e tecnologia avançada de produção. \\
Uma das principais entidades do setor de semicondutores no Brasil é a ABISEMI (Associação Brasileira da Indústria de Semicondutores), que tem como objetivo fomentar o crescimento e a competitividade da indústria nacional de semicondutores. A ABISEMI oferece suporte para empresas do setor, além de promover eventos e fóruns de discussão sobre a indústria de semicondutores no Brasil.\\
A CEITEC (Centro Nacional de Tecnologia Eletrônica Avançada) é uma empresa pública criada em 2008 para promover a produção nacional de semicondutores. A empresa tem como objetivo desenvolver tecnologias de ponta e incentivar a inovação tecnológica no país, com projetos de pesquisa e desenvolvimento de circuitos integrados para diversas aplicações, como identificação por radiofrequência (RFID) e smartcards.\\
A Intel, uma das maiores empresas de tecnologia do mundo, também tem investido em produção de semicondutores no Brasil. Em 2012, a empresa inaugurou uma fábrica em Santa Rita do Sapucaí, em Minas Gerais, para a produção de chips eletrônicos. A fábrica da Intel no Brasil é a primeira da empresa na América Latina e representa um importante avanço na consolidação da indústria nacional de semicondutores.\\
A Samsung, outra gigante da tecnologia, também tem investido no mercado de semicondutores no Brasil. A empresa possui uma fábrica em Campinas, São Paulo, que produz chips para dispositivos móveis e memórias flash NAND. A Samsung tem se destacado pela sua capacidade de inovação e pela qualidade de seus produtos, contribuindo para a competitividade da indústria nacional de semicondutores.\\
A Motorola, empresa americana de telecomunicações, também tem histórico de investimentos em produção de semicondutores no Brasil. Em 1998, a empresa inaugurou uma fábrica em Jaguariúna, São Paulo, para a produção de chips eletrônicos e outros componentes. Embora a empresa tenha enfrentado dificuldades financeiras nos últimos anos, seu histórico de investimentos e inovação na indústria de semicondutores no Brasil.\\
Com o crescente avanço tecnológico e a constante demanda por chips eletrônicos, a indústria de semicondutores é cada vez mais estratégica para o desenvolvimento econômico e tecnológico do Brasil. Investimentos em pesquisa, desenvolvimento e produção de semicondutores são essenciais para aumentar a competitividade da indústria nacional e promover a inovação tecnológica no país.\\
No entanto, a produção de semicondutores é um mercado altamente competitivo e dinâmico, com inúmeras empresas disputando espaço em um cenário global. Para manter a competitividade, o Brasil precisa continuar investindo em pesquisa e desenvolvimento de tecnologias de ponta, além de fomentar o empreendedorismo e a inovação em todas as etapas da cadeia de produção de semicondutores.\\
A indústria de semicondutores no Brasil tem muito potencial de crescimento, mas também enfrenta desafios, como a alta carga tributária, a falta de incentivos fiscais e a falta de mão de obra especializada. É importante que o governo e as empresas do setor trabalhem juntos para superar esses desafios e garantir que a indústria de semicondutores no Brasil continue a crescer e a contribuir para o desenvolvimento econômico e tecnológico do país.\\
Em resumo, a produção de semicondutores é um setor estratégico e em constante crescimento no Brasil, com empresas nacionais e internacionais investindo em pesquisa, desenvolvimento e produção de chips eletrônicos. Com o apoio do governo e da sociedade, a indústria de semicondutores tem potencial para se tornar uma importante força impulsionadora da economia e da inovação tecnológica no país.




% \begin{figure}[b]
%     \centering
%     \includegraphics[width=.6\textwidth]{aup.png}
%     \caption{Caption of your figure}
%     \label{fig:my_label}
% \end{figure}



  


% Include the body
% Note: You can also type it directly here, or include a file per section, whatever works best for you




% Bibliography

% Uncomment the next line (\nocite{*}) if you want to include all items from your .bib file
% (e.g. if you didn't use the \textcite or \parencite commands above)
 \nocite{*}

% This command generates the bibliography
\printbibliography


\end{document}

